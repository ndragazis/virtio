\section{Vhost-user Device Backend}\label{sec:Device Types / Vhost-user Device Backend}

The vhost-user device backend facilitates vhost-user device emulation through
vhost-user protocol exchanges and access to shared memory.  Software-defined
networking, storage, and other I/O appliances can provide services through this
device.

This section relies on definitions from the \hyperref[intro:Vhost-user
Protocol]{Vhost-user Protocol}.  Knowledge of the vhost-user protocol is a
prerequisite for understanding this device.

The \hyperref[intro:Vhost-user Protocol]{Vhost-user Protocol} was originally
designed for processes on a single system communicating over UNIX domain
sockets.  The virtio vhost-user device backend allows the vhost-user slave to
communicate with the vhost-user master over the device instead of a UNIX domain
socket.  This allows the slave and master to run on two separate systems such
as a virtual machine and a hypervisor.

The vhost-user slave program exchanges vhost-user protocol messages with the
vhost-user master through this device.  How the device implementation
communicates with the vhost-user master is beyond the scope of this
specification.  One possible device implementation uses a UNIX domain socket to
relay messages to a vhost-user master process running on the same host.

Existing vhost-user slave programs that communicate over UNIX domain sockets
can support the virtio vhost-user device backend without invasive changes
because the pre-existing vhost-user wire protocol is used.

\subsection{Device ID}\label{sec:Device Types / Vhost-user Device Backend / Device ID}
  28

\subsection{Virtqueues}\label{sec:Device Types / Vhost-user Device Backend / Virtqueues}

\begin{description}
\item[0] rxq (device-to-driver vhost-user protocol messages)
\item[1] txq (driver-to-device vhost-user protocol messages)
\end{description}

\subsection{Feature bits}\label{sec:Device Types / Vhost-user Device Backend / Feature bits}

No feature bits are defined at this time.

\subsection{Device configuration layout}\label{sec:Device Types / Vhost-user Device Backend / Device configuration layout}

  All fields of this configuration are always available.

\begin{lstlisting}
struct virtio_vhost_user_config {
        le32 status;
#define VIRTIO_VHOST_USER_STATUS_SLAVE_UP 0
#define VIRTIO_VHOST_USER_STATUS_MASTER_UP 1
        le32 max_vhost_queues;
        u8 uuid[16];
};
\end{lstlisting}

\begin{description}
\item[\field{status}] contains the vhost-user operational status.  The default
    value of this field is 0.

    The driver sets VIRTIO_VHOST_USER_STATUS_SLAVE_UP to indicate readiness for
    the vhost-user master to connect.  The vhost-user master cannot connect
    unless the driver has set this bit first.

    When the driver clears VIRTIO_VHOST_USER_SLAVE_UP while the vhost-user
    master is connected, the vhost-user master is disconnected.

    When the vhost-user master disconnects, both
    VIRTIO_VHOST_USER_STATUS_SLAVE_UP and VIRTIO_VHOST_USER_STATUS_MASTER_UP
    are cleared by the device.  Communication can be restarted by the driver
    setting VIRTIO_VHOST_USER_STATUS_SLAVE_UP again.

    A configuration change notification is sent when the device changes
    this field unless a write to the field by the driver caused the change.

\item[\field{max_vhost_queues}] is the maximum number of vhost-user queues
    supported by this device.  This field is always greater than 0.

\item[\field{uuid}] is the Universally Unique Identifier (UUID) for this
    device.  If the device has no UUID then this field contains the nil
    UUID (all zeroes).  The UUID allows vhost-user slave programs to identify a
    specific vhost-user device backend among many without relying on bus
    addresses.
\end{description}

\drivernormative{\subsubsection}{Device configuration layout}{Device Types / Vhost-user Device Backend / Device configuration layout}

The driver MUST NOT write to device configuration fields other than
\field{status}.

The driver MUST NOT set undefined bits in the \field{status} configuration field.

\devicenormative{\subsection}{Device Initialization}{Device Types / Vhost-user Device Backend / Device Initialization}

The driver SHOULD check the \field{max_vhost_queues} configuration field to
determine how many queues the vhost-user slave will be able to support.

The driver SHOULD fetch the \field{uuid} configuration field to allow
vhost-user slave programs to identify a specific device among many.

The driver SHOULD place at least one buffer in rxq before setting the
VIRTIO_VHOST_USER_SLAVE_UP bit in the \field{status} configuration field.

The driver MUST handle rxq virtqueue notifications that occur before the
configuration change notification.  It is possible that a vhost-user protocol
message from the vhost-user master arrives before the driver has seen the
configuration change notification for the VIRTIO_VHOST_USER_STATUS_MASTER_UP
\field{status} change.

\subsection{Device Operation}\label{sec:Device Types / Vhost-user Device Backend / Device Operation}

Device operation consists of operating request queues and response queues.

\subsubsection{Device Operation: Request Queues}\label{sec:Device Types / Vhost-user Device Backend / Device Operation / Device Operation: Request Queues}

The driver receives vhost-user protocol messages from the vhost-user master on
rxq.  The driver sends responses to the vhost-user master on txq.

The driver sends slave-initiated requests on txq.  The driver receives
responses from the vhost-user master on rxq.

All virtqueues offer in-order guaranteed delivery semantics for vhost-user
protocol messages.

Each buffer is a vhost-user protocol message as defined by the
\hyperref[intro:Vhost-user Protocol]{Vhost-user Protocol}.  In order to enable
cross-endian communication, all message fields are little-endian instead of the
native byte order normally used by the protocol.

The appropriate size of rxq buffers is at least as large as the largest message
defined by the \hyperref[intro:Vhost-user Protocol]{Vhost-user Protocol}
standard version that the driver supports.  If the vhost-user master sends a
message that is too large for an rxq buffer then DEVICE_NEEDS_RESET is set and
the driver must reset the device.

File descriptor passing is handled differently by the vhost-user device
backend.  When a message is received that carries one or more file descriptors
according to the vhost-user protocol, additional device resources become
available to the driver.

\subsection{Additional Device Resources over PCI}\label{sec:Device Types / Vhost-user Device Backend / Additional Device Resources over PCI}

The vhost-user device backend contains additional device resources beyond
configuration space and virtqueues.  The nature of these resources is
transport-specific and therefore only virtio transports that provide these
resources support the vhost-user device backend.

The following additional resources exist:
\begin{description}
  \item[Doorbells] The driver signals the vhost-user master through doorbells.  The signal does not carry any data, it is purely an event.
  \item[Notifications] The vhost-user master signals the driver for events besides virtqueue activity and configuration changes by sending notifications.
  \item[Shared memory] The vhost-user master gives access to memory that can be mapped by the driver.
\end{description}

\subsubsection{Doorbell Numbering}\label{sec:Device Types / Vhost-user Device Backend / Additional Device Resources over PCI / Doorbell Numbering}

Doorbells are laid out as follows:

\begin{description}
\item[0] Vring call for vhost-user queue 0
\item[\ldots]
\item[N] Vring err for vhost-user queue 0
\item[\ldots]
\item[2N] Log
\end{description}

\subsubsection{Notifications}\label{sec:Device Types / Vhost-user Device Backend / Additional Device Resources over PCI / Notifications}

Notifications are laid out as follows:

\begin{description}
\item[0] Vring kick for vhost-user queue 0
\item[\ldots]
\item[N-1] Vring kick for vhost-user queue N-1
\end{description}

\subsubsection{Shared Memory Layout}\label{sec:Device Types / Vhost-user Device Backend / Additional Device Resources over PCI / Shared Memory Layout}

Shared memory is laid out as follows:

\begin{description}
\item[0] Vhost memory region 0
\item[SIZE0] Vhost memory region 1
\item[\ldots]
\item[SIZE0 + SIZE1 + \ldots] Log
\end{description}

The size of vhost memory region 0 is \field{SIZE0}, the size of vhost memory
region 1 is \field{SIZE1}, and so on.

\subsubsection{Availability of Additional Resources}\label{sec:Device Types / Vhost-user Device Backend / Additional Device Resources over PCI / Availability of Additional Resources}

The following vhost-user protocol messages convey access to additional device
resources:

\begin{description}
\item[VHOST_USER_SET_MEM_TABLE] Contents of vhost memory regions are available to the driver in shared memory.  Region contents are laid out in the same order as the vhost memory region list.
\item[VHOST_USER_SET_LOG_BASE] Contents of the log are available to the driver in shared memory.
\item[VHOST_USER_SET_LOG_FD] The log doorbell is available to the driver.  Writes to the log doorbell before this message is received produce no effect.
\item[VHOST_USER_SET_VRING_KICK] The vring kick notification for this queue is available to the driver.  The first notification may occur before the driver has processed this message.
\item[VHOST_USER_SET_VRING_CALL] The vring call doorbell for this queue is available to the driver.  Writes to the vring call doorbell before this message is received produce no effect.
\item[VHOST_USER_SET_VRING_ERR] The vring err doorbell for this queue is available to the driver.  Writes to the vring err doorbell before this message is received produce no effect.
\item[VHOST_USER_SET_SLAVE_REQ_FD] The driver may send vhost-user protocol slave messages on txq.  Buffers put onto txq before this message is received are discarded by the device.
\end{description}

Additional resources are configured on the virtio PCI transport by the following \field{struct virtio_pci_cap.cfg_type} values:

\begin{lstlisting}
#define VIRTIO_PCI_CAP_DOORBELL_CFG 6
#define VIRTIO_PCI_CAP_NOTIFICATION_CFG 7
#define VIRTIO_PCI_CAP_SHARED_MEMORY_CFG 8
\end{lstlisting}

\subsubsection{Doorbell structure layout}\label{sec:Device Types / Vhost-user Device Backend / Additional Device Resources over PCI / Doorbell capability}

The doorbell location is found using the VIRTIO_PCI_CAP_DOORBELL_CFG
capability.  This capability is immediately followed by an additional
field, like so:

\begin{lstlisting}
struct virtio_pci_doorbell_cap {
        struct virtio_pci_cap cap;
        le32 doorbell_off_multiplier;
};
\end{lstlisting}

The doorbell address within a BAR is calculated as follows:

\begin{lstlisting}
        cap.offset + doorbell_idx * doorbell_off_multiplier
\end{lstlisting}

The \field{cap.offset} and \field{doorbell_off_multiplier} are taken from the
doorbell capability structure above, and the \field{doorbell_idx} is the
doorbell number.

\devicenormative{\paragraph}{Doorbell capability}{Device Types / Vhost-user Device Backend / Additional Device Resources over PCI / Doorbell capability}
The device MUST present at least one doorbell capability.

The \field{cap.offset} MUST be 2-byte aligned.

The device MUST either present \field{doorbell_off_multiplier} as an even power of 2,
or present \field{doorbell_off_multiplier} as 0.

The value \field{cap.length} presented by the device MUST be at least 2
and MUST be large enough to support doorbell offsets for all supported
doorbells in all possible configurations.

The value \field{cap.length} presented by the device MUST satisfy:
\begin{lstlisting}
cap.length >= num_doorbells * doorbell_off_multiplier + 2
\end{lstlisting}

The number of doorbells is \field{num_doorbells} and is dependent on the
device.

\subsubsection{Notification structure layout}\label{sec:Device Types / Vhost-user Device Backend / Additional Device Resources over PCI / Notification capability}

The notification structure allows MSI-X vectors to be configured for
notification interrupts.  If MSI-X is not available, bit 2 of the ISR status
indicates that a notification occurred.

The notification structure is found using the VIRTIO_PCI_CAP_NOTIFICATION_CFG
capability.

\begin{lstlisting}
struct virtio_pci_notification_cfg {
        le16 notification_select;              /* read-write */
        le16 notification_msix_vector;         /* read-write */
};
\end{lstlisting}

The driver indicates which notification is of interest by writing the
\field{notification_select} field.  The driver then writes the MSI-X vector or
\field{VIRTIO_MSI_NO_VECTOR} to \field{notification_msix_vector} to change the
MSI-X vector for that notification.

\devicenormative{\paragraph}{Notification capability}{Device Types / Vhost-user Device Backend / Additional Device Resources over PCI / Notification capability}

If MSI-X is available, device MUST support mapping any master queue
event to any valid vector 0 to MSI-X \field{Table Size}. Here
\field{Table Size} is the \field{N-1} encoded Table Size stored in the
Message Control register of the MSI-X capability structure according to
\hyperref[intro:PCI]{[PCI]}, where \field{N} is the actual MSI-X Table
Size.

Device MUST support unmapping any master queue event.

The device MUST return vector mapped to a given master queue event,
(\field{NO_VECTOR} if unmapped) on read of
\field{notification_msix_vector}. The device MUST have all master queue
events unmapped upon reset.

Devices SHOULD NOT cause mapping an event to vector to fail unless it is
impossible for the device to satisfy the mapping request. Devices MUST
report mapping failures by returning the \field{NO_VECTOR} value when
the relevant \field{notification_msix_vector} field is read.

\drivernormative{\paragraph}{Notification capability}{Device Types /
Vhost-user Device Backend / Additional Device Resources over PCI /
Noification capability}

Driver MUST support device with any MSI-X Table Size 0 to 0x7FF. Driver
MAY fall back on using INT\#x interrupts for a device which only
supports one MSI-X vector (MSI-X Table Size = 0).

Driver MAY intepret the \field{Table Size} as a hint from the device for
the suggested number of MSI-X vectors to use.

Driver MUST NOT attempt to map an event to a vector outside the MSI-X
Table supported by the device, as reported by \field{Table Size} in the
MSI-X Capability.

After mapping an event to vector, the driver MUST verify success by
reading the Vector field value: on success, the previously written value
is returned, and on failure, \field{NO_VECTOR} is returned. If a mapping
failure is detected, the driver MAY retry mapping with fewer vectors,
disable MSI-X or report device failure.

\subsubsection{Shared memory capability}\label{sec:Device Types / Vhost-user Device Backend / Additional Device Resources over PCI / Shared Memory capability}

The shared memory location is found using the VIRTIO_PCI_CAP_SHARED_MEMORY_CFG
capability.

\devicenormative{\paragraph}{Shared Memory capability}{Device Types / Vhost-user Device Backend / Additional Device Resources over PCI / Shared Memory capability}
The device MUST present exactly one shared memory capability.

The device MUST locate shared memory in a Memory Space BAR.

The device SHOULD locate shared memory in a Prefetchable BAR.

The \field{cap.offset} MUST be 4096-byte aligned.

The value \field{cap.length} presented by the device MUST be non-zero and 4096-byte aligned.
