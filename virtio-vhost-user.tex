\section{Vhost-user Device Backend}\label{sec:Device Types / Vhost-user Device Backend}

The vhost-user device backend facilitates vhost-user device emulation through
vhost-user protocol exchanges and access to shared memory.  Software-defined
networking, storage, and other I/O appliances can provide services through this
device.

This section relies on definitions from the \hyperref[intro:Vhost-user
Protocol]{Vhost-user Protocol}.  Knowledge of the vhost-user protocol is a
prerequisite for understanding this device.

The \hyperref[intro:Vhost-user Protocol]{Vhost-user Protocol} was originally
designed for processes on a single system communicating over UNIX domain
sockets.  The virtio vhost-user device backend allows the vhost-user slave to
communicate with the vhost-user master over the device instead of a UNIX domain
socket.  This allows the slave and master to run on two separate systems such
as a virtual machine and a hypervisor.

The vhost-user slave program exchanges vhost-user protocol messages with the
vhost-user master through this device.  How the device implementation
communicates with the vhost-user master is beyond the scope of this
specification.  One possible device implementation uses a UNIX domain socket to
relay messages to a vhost-user master process running on the same host.

Existing vhost-user slave programs that communicate over UNIX domain sockets
can support the virtio vhost-user device backend without invasive changes
because the pre-existing vhost-user wire protocol is used.

\subsection{Device ID}\label{sec:Device Types / Vhost-user Device Backend / Device ID}
  32

\subsection{Virtqueues}\label{sec:Device Types / Vhost-user Device Backend / Virtqueues}

\begin{description}
\item[0] rxq (device-to-driver vhost-user protocol messages)
\item[1] txq (driver-to-device vhost-user protocol messages)
\end{description}

\subsection{Feature bits}\label{sec:Device Types / Vhost-user Device Backend / Feature bits}

No feature bits are defined at this time.

\subsection{Device configuration layout}\label{sec:Device Types / Vhost-user Device Backend / Device configuration layout}

All fields of this configuration are always available.

\begin{lstlisting}
struct virtio_vhost_user_config {
        le32 status;
#define VIRTIO_VHOST_USER_STATUS_SLAVE_UP (1 << 0)
#define VIRTIO_VHOST_USER_STATUS_MASTER_UP (1 << 1)
        le32 max_vhost_queues;
        u8 uuid[16];
};
\end{lstlisting}

\begin{description}
\item[\field{status}] contains the vhost-user operational status.  The default
    value of this field is 0.

    The driver sets VIRTIO_VHOST_USER_STATUS_SLAVE_UP to indicate readiness for
    the vhost-user master to connect.  The vhost-user master cannot connect
    unless the driver has set this bit first.

    The device sets VIRTIO_VHOST_USER_STATUS_MASTER_UP to indicate that the
    vhost-user master is connected.

    When the driver clears VIRTIO_VHOST_USER_STATUS_SLAVE_UP while the
    vhost-user master is connected, the vhost-user master is disconnected.

    When the vhost-user master disconnects, both
    VIRTIO_VHOST_USER_STATUS_SLAVE_UP and VIRTIO_VHOST_USER_STATUS_MASTER_UP
    are cleared by the device.  Communication can be restarted by the driver
    setting VIRTIO_VHOST_USER_STATUS_SLAVE_UP again.

    A configuration change notification is sent when the device changes
    this field, unless a write to the field by the driver caused the change.

\item[\field{max_vhost_queues}] is the maximum number of vhost-user queues
    supported by this device.  This field is always greater than 0.

\item[\field{uuid}] is the Universally Unique Identifier (UUID) for this
    device.  If the device has no UUID, then this field contains the nil
    UUID (all zeroes).  The UUID allows vhost-user slave programs to identify a
    specific vhost-user device backend among many without relying on bus
    addresses.
\end{description}

\drivernormative{\subsubsection}{Device configuration layout}{Device Types / Vhost-user Device Backend / Device configuration layout}

The driver MUST NOT write to device configuration fields other than
\field{status}.

The driver MUST NOT set undefined bits in the \field{status} configuration field.

\subsection{Device Initialization}\label{sec:Device Types / Vhost-user Device Backend / Device Initialization}

The driver initializes the rxq/txq virtqueues and then it sets
VIRTIO_VHOST_USER_STATUS_SLAVE_UP to the \field{status} field of the device
configuration structure.

\drivernormative{\subsubsection}{Device Initialization}{Device Types / Vhost-user Device Backend / Device Initialization}

The driver SHOULD check the \field{max_vhost_queues} configuration field to
determine how many queues the vhost-user slave will be able to support.

The driver SHOULD fetch the \field{uuid} configuration field to allow
vhost-user slave programs to identify a specific device among many.

The driver SHOULD place at least one buffer in rxq before setting the
VIRTIO_VHOST_USER_STATUS_SLAVE_UP bit in the \field{status} configuration field.

The driver MUST handle rxq virtqueue notifications that occur before the
configuration change notification.  It is possible that a vhost-user protocol
message from the vhost-user master arrives before the driver has seen the
configuration change notification for the VIRTIO_VHOST_USER_STATUS_MASTER_UP
\field{status} change.

\subsection{Device Operation}\label{sec:Device Types / Vhost-user Device Backend / Device Operation}

Device operation consists of operating request queues and response queues.

\subsubsection{Device Operation: RX/TX Queues}\label{sec:Device Types / Vhost-user Device Backend / Device Operation / Device Operation: RX/TX Queues}

The driver receives vhost-user protocol messages from the vhost-user master on
rxq.  The driver sends responses to the vhost-user master on txq.

The driver sends slave-initiated requests on txq.  The driver receives
responses from the vhost-user master on rxq.

All virtqueues offer in-order guaranteed delivery semantics for vhost-user
protocol messages.

Each buffer is a vhost-user protocol message as defined by the
\hyperref[intro:Vhost-user Protocol]{Vhost-user Protocol}.  In order to enable
cross-endian communication, all message fields are little-endian instead of the
native byte order normally used by the protocol.

The appropriate size of rxq buffers is at least as large as the largest message
defined by the \hyperref[intro:Vhost-user Protocol]{Vhost-user Protocol}
standard version that the driver supports.  If the vhost-user master sends a
message that is too large for an rxq buffer, then DEVICE_NEEDS_RESET is set and
the driver must reset the device.

File descriptor passing is handled differently by the vhost-user device
backend.  When a message is received that carries one or more file descriptors
according to the vhost-user protocol, additional device resources become
available to the driver.

\subsection{Additional Device Resources}\label{sec:Device Types / Vhost-user Device Backend / Additional Device Resources}

The vhost-user device backend contains additional device resources, except for
those specificied in \ref{sec:Basic Facilities of a Virtio Device}. The
additional device resources allow the vhost-user master and slave to exchange
notifications and data through the device.

The following additional resources exist:
\begin{description}
  \item[Doorbells] The driver signals the vhost-user master through doorbells.  The signal does not carry any data, it is purely an event.
  \item[Notifications] The vhost-user master signals the driver for events besides virtqueue activity and configuration changes by sending notifications.
  \item[Shared memory] The vhost-user master gives access to memory that can be mapped by the driver.
\end{description}

\subsubsection{Doorbells}\label{sec:Device Types / Vhost-user Device Backend / Additional Device Resources / Doorbells}

The vhost-user device backend provides all (or part) of the following doorbells:

\begin{description}
\item[0] Vring call for vhost-user queue 0
\item[\ldots]
\item[N-1] Vring call for vhost-user queue N-1
\item[N] Vring err for vhost-user queue 0
\item[\ldots]
\item[2N-1] Vring err for vhost-user queue N-1
\item[2N] Log
\end{description}

where N is the number of the vhost-user virtqueues.

\subsubsection{Notifications}\label{sec:Device Types / Vhost-user Device Backend / Additional Device Resources / Notifications}

The vhost-user device backend provides all (or part) of the following
notifications:

\begin{description}
\item[0] Vring kick for vhost-user queue 0
\item[\ldots]
\item[N-1] Vring kick for vhost-user queue N-1
\end{description}

where N is the number of the vhost-user virtqueues.

\subsubsection{Shared Memory}\label{sec:Device Types / Vhost-user Device Backend / Additional Device Resources / Shared Memory}

The vhost-user device backend provides all (or part) of the following shared
memory regions:

\begin{description}
\item[0] Vhost-user memory region 0
\item[1] Vhost-user memory region 1
\item[\ldots]
\item[M-1] Vhost-user memory region M-1
\item[M] Log memory region
\end{description}

\subsubsection{Availability of Additional Resources}\label{sec:Device Types / Vhost-user Device Backend / Additional Device Resources / Availability of Additional Resources}

The following vhost-user protocol messages convey access to additional device
resources:

\begin{description}
\item[VHOST_USER_SET_MEM_TABLE] Contents of vhost-user memory regions are available to the driver as device memory. Region contents are laid out in the same order as the vhost-user memory region list.
\item[VHOST_USER_SET_LOG_BASE] Contents of the log memory region are available to the driver as device memory.
\item[VHOST_USER_SET_LOG_FD] The log doorbell is available to the driver.  Writes to the log doorbell before this message is received produce no effect.
\item[VHOST_USER_SET_VRING_KICK] The vring kick notification for this queue is available to the driver.  The first notification may occur before the driver has processed this message.
\item[VHOST_USER_SET_VRING_CALL] The vring call doorbell for this queue is available to the driver.  Writes to the vring call doorbell before this message is received produce no effect.
\item[VHOST_USER_SET_VRING_ERR] The vring err doorbell for this queue is available to the driver.  Writes to the vring err doorbell before this message is received produce no effect.
\item[VHOST_USER_SET_SLAVE_REQ_FD] The driver may send vhost-user protocol slave messages on txq.  Buffers put onto txq before this message is received are discarded by the device.
\end{description}

\subsubsection{Doorbell layout}\label{sec:Device Types / Vhost-user Device Backend / Additional Device Resources / Doorbell layout}

The device MUST reserve 2N+1 virtqueue indices that can be used by the driver to
send doorbell notifications. The driver can use these indices to send doorbell
notifications in the same way that it sends available buffer notifications
\ref{sec:Basic Facilities of a Virtio Device / Notifications} for a virtqueue.

Specifically, the driver can find the locations and send notifications on the
call and err doorbells by using the following special queue indices:

\begin{itemize}
\item 2i+2: find call doorbell for vhost-user queue i
\item 2i+3: find err doorbell for vhost-user queue i
\end{itemize}

where i is the vhost-user queue index (defined by the vhost-user master via the
vhost-user messages). The indices 0 and 1 are reserved for the device's RX/TX
virtqueues.

For the Log doorbell, the driver can use the special queue index 2(N+1), where N
is the number of vhost-user queues.

\subsubsection{Notification layout}\label{sec:Device Types / Vhost-user Device Backend / Additional Device Resources / Notification layout}

If MSI-X is available, the driver can specify an MSI-X vector for a vhost-user
queue in the same way that it specifies an MSI-X vector for a virtqueue. The
driver MUST use the special index 2i+2 to specify an MSI-X vector for the
vhost-user queue with index i. The vhost-user queue indices are defined by the
vhost-user master via the vhost-user messages.

If MSI-X is not available, bit 2 (i.e., the third least significant bit) of the
ISR status register indicates that a notification occured for one of the
vhost-user queues.

\subsubsection{Shared Memory layout}\label{sec:Device Types / Vhost-user Device Backend / Additional Device Resources / Shared Memory layout}

The device exports all memory regions reported by the vhost-user master as a
single shared memory region \ref{sec:Basic Facilities of a Virtio Device /
Shared Memory Regions}.

The size of this shared memory region MUST be at least as much as the sum of the
sizes of all the memory regions reported by the vhost-user master.

The memory regions MUST be laid out in the same order in which they are reported
by the master with vhost-user messages.

The offsets in which the memory regions are mapped inside the shared memory
region MUST be the following:

\begin{description}
\item[0] Offset for vhost-user memory region 0
\item[SIZE0] Offset for vhost-user memory region 1
\item[\ldots]
\item[SIZE0 + SIZE1 + \ldots] Offset for vhost-user memory region M
\end{description}

where SIZEi is the size of the vhost-user memory region i.
